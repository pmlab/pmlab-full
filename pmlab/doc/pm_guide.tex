\documentclass[a4paper,10pt]{book}
\usepackage[margin=2.5cm]{geometry}

\usepackage{ucs}
\usepackage[utf8]{inputenc}
\usepackage{amsmath}
\usepackage{amsfonts}
\usepackage{amssymb}
\usepackage[american]{babel}
\usepackage{fontenc}
\usepackage{graphicx}
\usepackage{minted}
\usepackage{hyperref}

\usepackage{fancyvrb}
\usepackage{color}
\usepackage{dirtree}

\input{pythoncode.tex}

\title{PM user guide}
\author{Marc Solé}
\date{25-09-12}

\begin{document}
 \maketitle

\tableofcontents

\chapter{Introduction}
\section{Motivation}
After a couple of years researching in Process Mining, I ended up with a set of loosely related applications programmed in C++ and Python. The set of applications and scripts was working quite well, but the communication between them was tedious. In 2012 I also discovered IPython (\url{http://ipython.org/})and thought that this was the perfect ecosystem in which a Python package for Process Mining could do wonders.

Why? Well, Process Mining (like Formal Verification, which is my background) is currently more an art than a science (although both areas involve enormous amounts of science). This is something that anybody working with real datasets can tell you. Typically you have to look at the data in several ways and then figure out which type of processing would yield the best results for what you intent to achieve. This requires a high-degree of flexibility and an environment in which one can quickly test ideas.

Such exploration process is perfectly suited for IPython/Python capabilities. There are other Process Mining suites, notably the excellent ProM~\cite{AalstDGMMRRSVW07} (\url{http://www.promtools.org/prom6/}) with all its plug-ins. ProM is fantastic as a user, but requires quite an effort to become a developer. Also I always found it difficult to execute things in batch and, of course, it does not provide you with the ability to literally play with data, as this suite tries to do. Since we offer a programming environment, at any point you can do whatever you are able to program. And in Python you can program succintly quite a lot of things.

I hope you will enjoy this package as much as I enjoy it myself. Happy coding and happy mining!

\section{Log extraction}
The typical problem one has to face when performing process mining on real data is obtaining such data (log). There is specialized software for this task (for instance, Nitro~\footnote{Available at \url{http://www.fluxicon.com/nitro/}.}), but in my experience I frequently end up writing some Python script to generate a log. Let me show an example.

Consider the following data extracted from a recommendation software of Chicago restaurants:
\begin{verbatim}
07/Sep/1996:12:17:05 	www-c09.proxy.gnn.com	0	560L	110L	110
07/Sep/1996:12:21:57 	foley.ripco.com	0	423L	77L	77
07/Sep/1996:12:23:36 	foley.ripco.com	0	633N	633N	159P	159P	475
07/Sep/1996:12:25:14 	foley.ripco.com	0	441L	537L	537
...
\end{verbatim}
Here the actions we are interested in are the letters that appear from the fourth column until the previous to the last. Each line represents a sequence of user actions. To obtain a log from that data, I used the following Python script, to whom the filename of the file containing the data is passed as a parameter:
\definecolor{bg}{rgb}{0.95,0.95,0.95}
\newminted{py}{bgcolor=bg}
\begin{pycode}
"""script to generate events from the recommendation system example"""
import sys

with open(sys.argv[1]) as f:
    for line in f:
	words = line.split()
	for w in words[3:-1]:
	    print w[-1],
	print
\end{pycode}

Executing this script with the previous file, we obtain the following output, which can be redirected to a file. 
\begin{verbatim}
L L
L L
N N P P
L L
...
\end{verbatim}
The resulting file is in ``raw'' format (see more on this format in \ref{sec:log.load}).

\section{Organization of the Package}

A Python package strictly follows a directory structure and filename convention.
\dirtree{%
.1 pm.
.2 \_\_init\_\_.py.
.2 log.
.3 \_\_init\_\_.py.
.3 reencoders.py.
.3 projectors.py.
.3 filters.py.
.3 clustering.py.
.3 noise.py.
.2 ts.
.3 \_\_init\_\_.py.
.2 pn.
.3 \_\_init\_\_.py.
.3 simulation.py.
.2 cnet.
.3 \_\_init\_\_.py.
.2 bpmn.
.3 \_\_init\_\_.py.
}

A package is a directory containing a \texttt{\_\_init\_\_.py} file, and a module is a \texttt{*.py} file (other than \texttt{\_\_init\_\_.py}). As one can deduce from the directory structure, a package can contain other packages as well as modules. The importing mechanism for both elements is the same. The statement:\\
\begin{pycode}
import pm
\end{pycode}
will make available to the user all the objects defined inside the \texttt{\_\_init\_\_.py} file of the \texttt{pm} directory. Similarly, writing:\\
\begin{pycode}
import pm.log.filters
\end{pycode}
will load all the objects inside the \texttt{filters.py} file in the \texttt{pm/log} directory.

\section{Downloading}

\section{License}
Since some of the software we use is GPL, the license of the suite is GPL v3.

\section{Installation}
Currently the package assumes that the following applications are available:
\begin{itemize}
 \item rbminer (\url{http://people.ac.upc.edu/msole/homepage/Software.html})
 \item log2ts (available at the same address as rbminer)
 \item stp (\url{http://stp.github.io/}) [MIT]
\end{itemize}
It assumes they are located in \texttt{/usr/local/bin}. If some of these applications are not installed, the package will work except for the functionalities that require these applications (which are quite a few).

Besides that, the package also depends on some python packages. In particular:
\begin{itemize}
 \item graph\_tool (\url{http://projects.skewed.de/graph-tool/}) [GPL]
 \item pyparsing [Free license]
 \item pygame [LGPL]
\end{itemize}
However, these packages are already defined in the setup script of our package, so that typing:
\begin{verbatim}
python setup.py install 
\end{verbatim}
should already fetch and install the required packages (if not already present in the system). In some cases it might be advisable to install these Python packages using the installers provided by the Linux distribution (i.e. through synaptic or another package manager). This is particularly true for packages that include C code such as graph\_tool and pygame, since they usually have they own dependencies and the Linux package system simplifies the task. 

\section{IPython basics}

We strongly recommend the use of IPython to work with this package. We summarize some of the most useful additional commands it provides with respect to the standard Python interpreter.

First of all, IPython can execute the same commands as the standard Python interpreter. However you can check the current directory with the \texttt{pwd} command, list it with \texttt{ls}, copy files with \texttt{cp} and execute arbitrary shell commands by prefixing the command with the \texttt{!} symbol.

IPython will store all commands typed, so that they can be retrieved afterwards. For instance 
\begin{verbatim}
history ~3/1-~0/200
\end{verbatim}
will print from the first command of 3 sessions ago to the command 200 of the current session. Commands can be saved to a file using the \texttt{save} command or rerun (\texttt{rerun}).

A very helpful feature of IPython for a package half documented like this one, is the help function. We can see the docstring of any module or variable by simply writing the name of the module/variable (the module must have been imported before, and the variable must exist, obviously) and appending \texttt{?}. We can see the source code by appending \texttt{??}, which is a very nice feature to find interesting methods in the classes that are not described in this document.

\chapter{Logs}
To work with logs, the log module has to be imported:\\
\begin{pycode}
import pm.log  
\end{pycode}

\section{Plain logs and Enhanced logs}
The package was developed basically for plain logs, that is a list of activity sequences, in which an activity is simply a string (activity name). Log objects provide two basic methods to access these sequences:
\begin{description}
 \item [get\_cases] Returns the list of all cases (with repetitions).
 \item [get\_uniq\_cases] Returns a dictionary in which each unique case is mapped to the number of occurrences in the log.
\end{description}

An example, in which we first create a log by providing all the cases, and then we print the unique cases:
%\begin{minted}{python}
\begin{pycode}
>>> log = pm.log.Log(cases=[['A','B','A','B'],['C','D'],['A','C','B'],['C','D']])
>>> log.get_uniq_cases()
defaultdict(<type 'int'>, {('A', 'C', 'B'): 1, ('C', 'D'): 2, ('A', 'B', 'A', 'B'): 1})
\end{pycode}
%\end{minted} 


In this other example we do it the other way round: the log is created specifying occurrences (note that the sequences need to be a tuple then not a list, since dictionary keys must be non-mutable) and we print all the cases:\\
%\begin{minted}{python}
\begin{pycode}
>>> log = pm.log.Log(uniq_cases={('A','B','A'):2,('C','D'):3,('E',):1})
>>> log.get_cases()
[('A', 'B', 'A'), ('A', 'B', 'A'), ('C', 'D'), ('C', 'D'), ('C', 'D'), ('E',)]
\end{pycode}
%\end{minted}

Note that singleton tuples need a final comma (otherwise they are interpreted as normal parenthesis).

Enhanced logs, on the other hand, provide a general mechanism to add arbitrary information to each activity, using a dictionary. In this example we define an enhanced log and we obtain its cases. Note that the default behavior of the \texttt{get\_cases} function is to return the activities names (it uses the \emph{name} field of the dictionary) so that algorithms working on plain logs will work also for enhanced logs without modification (beware that, however in many cases the result would be a plain log). If all the information want to be retrieved then we must provide a parameter to the function.\\
%\begin{minted}{python} 
\begin{pycode}
>>> log = pm.log.EnhancedLog(cases=[[{'name':'A','timestamp':1},{'name':'B','timestamp':3}],
				    [{'name':'E','timestamp':5}]])
>>> log.get_cases()
[['A', 'B'], ['E']]
>>> log.get_cases(True)
[[{'name': 'A', 'timestamp': 1}, {'name': 'B', 'timestamp': 3}],
 [{'name': 'E', 'timestamp': 5}]]
\end{pycode}
%\end{minted}

\section{Manual modification of logs}
\label{sec:log.manual}

There are two things that must be taken into account when modifying/adding/removing cases of a log. One is that Python uses references (so in general assignment does not produce a copy of the data), the other is that logs memoize cases/unique cases, so that if a log is created by providing the unique cases, the first time that all the cases are demanded they are computed and stored. Further modification of unique cases will not trigger the recomputation of all the cases. For this reasons the safe way to modify a log is simply to create a new one by selecting the appropriate cases.

Let us see an example:\\
%\begin{minted}{python}
\begin{pycode}
>>> log = pm.log.Log(cases=[['A','B'],['C','D']])
>>> log.get_cases()
[['A', 'B'], ['C', 'D']]
>>> import copy
>>> cases = copy.deepcopy(log.get_cases())
>>> cases
[['A', 'B'], ['C', 'D']]
>>> del cases[1][1]
>>> cases
[['A', 'B'], ['C']]
>>> cases[0][0] = 'S'
>>> cases
[['S', 'B'], ['C']]
>>> cases[0].append('E')
>>> cases
[['S', 'B', 'E'], ['C']]
>>> nlog = pm.log.Log(cases=cases)
>>> nlog.get_uniq_cases()
defaultdict(<type 'int'>, {('C',): 1, ('S', 'B', 'E'): 1})
>>> log.get_uniq_cases()
defaultdict(<type 'int'>, {('A', 'B'): 1, ('C', 'D'): 1})
\end{pycode}
%\end{minted}

As you can see the cases between log and nlog are not shared. This is thanks to the deep copy of the cases performed by the standard copy module. Should we have not deep copied the cases, the modifications would have been performed to both logs (since the cases would be shared).

For instructions on how elements can be inserted in arbitrary positions in a Python list, please consult the Python documentation, since the syntax is quite powerful but out of scope of this document.

In the following example we construct a log by merging the two previous logs. To show the effects of the inherent sharing of Python, in this case we intentionally forgot the deep copy:\\
%\begin{minted}{python}
\begin{pycode}
>>> slog = pm.log.Log(cases=log.get_cases()+nlog.get_cases())
>>> slog.get_cases()
[['A', 'B'], ['C', 'D'], ['S', 'B', 'E'], ['C']]
>>> slog.get_cases()[0][0] = 'Z'
>>> slog.get_cases()
[['Z', 'B'], ['C', 'D'], ['S', 'B', 'E'], ['C']]
>>> log.get_cases()
[['Z', 'B'], ['C', 'D']]
\end{pycode}
%\end{minted}
In this case, the modification of the merged log entails the modification of one of the component logs.

For the particular operation of merging, the log class overrides the sum operator that performs a deep copy of the cases:\\
%\begin{minted}{python}
\begin{pycode}
>>> alog = log + nlog
>>> alog.get_cases()
[['A', 'B'], ['C', 'D'], ['S', 'B', 'E'], ['C']]
\end{pycode}
%\end{minted}

A final word of warning: the current implementation tries to save as much as possible writing temporal files for the applications that work with files. This means that if a log is loaded from a file and no modification is performed, then the original file is used instead of writing a temporal copy of it. All log methods that modify the log keep track of that, however, it is not possible for the class to keep track of the user manual modifications. The best approach is to generate a new log rather than performing inplace changes.

To sum up. To perform manual changes follow these steps (in that order):
\begin{enumerate}
 \item Deep copy the (unique) cases to a variable.
 \item Perform the desired modifications using all the power of Python syntax.
 \item Create a new log from the modified (unique) cases.
\end{enumerate}

\section{Loading/Saving}
\label{sec:log.load}
To load a log from a file we use the following function:\\
\begin{pycode}
log = pm.log.log_from_file('96q3.tr')
\end{pycode}

One of the parameters available is the \emph{format} parameter. If nothing is specified, then the filename extension is used to try to infer the file format. In this case, the '.tr' extension refers to a file in ``raw'' format. Support for XES files is also provided. Note that the default \emph{xes} option will only consider the activity names. If all the information wants to be stored, use the \emph{xes\_all} format option, that returns an enhanced log.

The raw format is just a list of sequences of activity names separated by some whitespace(s). Sequences are separated by newlines. Activity names cannot contain whitespaces, even if quotes are used. Several additional parameters are available, please check the docstring and the next example.

If the log is already stored in some iterable, e.g., a list of strings, then the function \mint{python}|pm.log.log_from_iterable|
can be used instead, as in this example:\\
\begin{pycode}
>>> test2 = ['# Reading file: a12f0n00_1.filter.tr', '.state graph', 's0 S s1']
>>> log2 = pm.log.log_from_iterable(test2,
				    reencoder=lambda x: x.translate(None,'.'),
				    comment_marks='#')
[['state', 'graph'], ['s0', 'S', 's1']]
\end{pycode}

% \begin{Verbatim}[commandchars=\\\{\}]
% \PY{o}{>>}\PY{o}{>} \PY{n}{test2} \PY{o}{=} \PY{p}{[}\PY{l+s}{'}\PY{l+s}{\PYZsh{} Reading file: a12f0n00\PYZus{}1.filter.tr}\PY{l+s}{'}\PY{p}{,} \PY{l+s}{'}\PY{l+s}{.state graph}\PY{l+s}{'}\PY{p}{,} \PY{l+s}{'}\PY{l+s}{s0 S s1}\PY{l+s}{'}\PY{p}{]}
% \PY{o}{>>}\PY{o}{>} \PY{n}{log2} \PY{o}{=} \PY{n}{pm}\PY{o}{.}\PY{n}{log}\PY{o}{.}\PY{n}{log\PYZus{}from\PYZus{}iterable}\PY{p}{(}\PY{n}{test2}\PY{p}{,}
%                                     \PY{n}{reencoder}\PY{o}{=}\PY{k}{lambda} \PY{n}{x}\PY{p}{:} \PY{n}{x}\PY{o}{.}\PY{n}{translate}\PY{p}{(}\PY{n+nb+bp}{None}\PY{p}{,}\PY{l+s}{'}\PY{l+s}{.}\PY{l+s}{'}\PY{p}{)}\PY{p}{,} 
%                                     \PY{n}{comment\PYZus{}marks}\PY{o}{=}\PY{l+s}{'}\PY{l+s}{\PYZsh{}}\PY{l+s}{'}\PY{p}{)}    
% \PY{p}{[}\PY{p}{[}\PY{l+s}{'}\PY{l+s}{state}\PY{l+s}{'}\PY{p}{,} \PY{l+s}{'}\PY{l+s}{graph}\PY{l+s}{'}\PY{p}{]}\PY{p}{,} \PY{p}{[}\PY{l+s}{'}\PY{l+s}{s0}\PY{l+s}{'}\PY{p}{,} \PY{l+s}{'}\PY{l+s}{S}\PY{l+s}{'}\PY{p}{,} \PY{l+s}{'}\PY{l+s}{s1}\PY{l+s}{'}\PY{p}{]}\PY{p}{]}
% \end{Verbatim}
Here we considered as comments all the lines starting with '\#', and all activity names are encoded with a function that removes all dots inside the activity names.

To save a log, use the save function of a log object:\\
\begin{pycode}
log.save('test.tr')
\end{pycode}

Currently we provide support for saving in raw format and in XES format.

\section{Encoders}
The purpose of encoders is to adapt (i.e., change) the activity names of the logs, typically to make them compliant for some third-party tool. There are several already ready-to-use encoders, as well as a general mechanism to load/save dictionary-based encoders. In any case creating an encoder should be quite easy. To import the encoders type:\\
\begin{pycode}
import pm.log.reencoders
\end{pycode}

\subsection{Default encoders}
The are three classes of encoders already available:
\begin{description}
 \item [AlphaReencoder] Encodes each activity using single characters, starting by 'a'. Nice to save space in drawings if semantics of activities are not specially important.
 \item [EventNumberReencoder] Encodes each activity using 'e' plus a number. Works better than the AlphaReencoder if there are lots of activities (since then the characters can become non-alphabetical and even non-ASCII).
 \item [StpReencoder] Removes problematic characters when using the STP SMT-solver.
\end{description}

\subsection{Loading and saving encoders}
All encoders inherit from the DictionaryEncoder, that provides a save function to store the translation dictionary into a file. Conversely there is a function to load the generated files, as in this example:\\
\begin{pycode}
>>> ne.dict
{'a': 'e0', 'b': 'e1', 'c': 'e2', 'd': 'e3', 'e': 'e4'}
>>> ne.save('test.dict')
>>> e = pm.log.reencoders.reencoder_from_file('test.dict')
>>> e.dict
{'a': 'e0', 'b': 'e1', 'c': 'e2', 'd': 'e3', 'e': 'e4'}
\end{pycode}

\subsection{Encoding a log}
A log can be encoded during load from a file passing the encoder as a parameter to the loading function, or just calling the reencode function of a log object:\\
\begin{pycode}
log.reencode( pm.log.reencoders.AlphaReencoder() )
\end{pycode}

Please note that if using the encoder during the log load, the parameter expects an encoding function, not an object, so you must provide a reference to the \texttt{reencode} function of the encoder. However you can just use a lambda function. In any case, the resulting DictionaryReencoder is stored inside the log (\emph{reencoder\_on\_load} field), so that the process is reversible even if a lambda function is used.

It is possible to obtain an inverted encoder (i.e., a decoder) for a given encoder, so that one can restore a log after it has been processed, as in the next example:\\
\begin{pycode}
>>> log.get_cases()
[['a', 'b', 'c'], ['a', 'd', 'e', 'c'], ['a', 'e', 'd', 'c']]
>>> ne = pm.log.reencoders.EventNumberReencoder()
>>> log.reencode(ne)
>>> log.get_cases()
[['e0', 'e1', 'e2'], ['e0', 'e3', 'e4', 'e2'], ['e0', 'e4', 'e3', 'e2']]
>>> ie = ne.decoder()
>>> ie.dict
{'e0': 'a', 'e1': 'b', 'e2': 'c', 'e3': 'd', 'e4': 'e'}
>>> log.reencode(ie)
>>> log.get_cases()
[['a', 'b', 'c'], ['a', 'd', 'e', 'c'], ['a', 'e', 'd', 'c']]
\end{pycode}

Note that the reencode function of a log has parameters to automatically save the encoder to a file, saving some typing (consult the docstring of the function).

\section{Projections}
The purpose of projectors is to remove/keep some of the activities in the logs (vertical slicing). However this module also allows for some horizontal slicing considering the activities that appear in each case. To import the projectors type:\\
\begin{pycode}
import pm.log.projectors
\end{pycode}

This module contains one main functions and several helping ones. The main function is \texttt{project\_log}. As shown in this example its default usage is to preserve only the activities passed as parameters:\\
\begin{pycode}
>>> log = pm.log.Log(cases=[['A','B'],['C','D']])
>>> plog = pm.log.projectors.project_log(log, ['A','D'])
>>> plog.get_cases()
[['A'], ['D']]
\end{pycode}

However, using the parameter \emph{action} the behavior can be modified. For instance, we can specify the list of activities to remove (action='suppress') or that the case is kept or removed only if all the activities are/are not in the list (actions \emph{keep\_if\_all} and \emph{suppress\_if\_all}). To complete all the logical possibilities we have the additional actions \emph{keep\_if\_any} and \emph{suppress\_if\_any} that will keep/remove a case if at least one of its activities appears in the list.

The helper functions in this module are used to build reasonable activity lists, based on the activity frequencies. In the following example, we will select the two most frequent activities and project the log, keeping only the cases containing some of these activities.\\
\begin{pycode}
>>> log = pm.log.Log(cases=[['A','B','A','B'],['C','D'],['A','C','B']])
>>> freq = log.activity_frequencies()
>>> freq
defaultdict(<type 'int'>, {'A': 3, 'C': 2, 'B': 3, 'D': 1})
>>> mostfreq = pm.log.projectors.most_frequent(freq, 2)
>>> mostfreq
{'A': 3, 'B': 3}
>>> plog = pm.log.projectors.project_log(log, mostfreq, action='keep_if_any')
>>> plog.get_cases()
[['A', 'B', 'A', 'B'], ['A', 'C', 'B']]
\end{pycode}

The other helper function \texttt{above\_threshold} simply keeps all the activities whose frequency is above the given threshold. Note that ``activity frequency'' in general refers to the total number of occurrences in all the log. However, the \texttt{activity\_frequencies} function can also give the number of cases in which an activity appears (this is useful to give a percentage of occurrence, which is the notion of ``activity frequency'' used in ProM).

\section{Filters}
As with projections, the filter module has one main function and several already-defined filters. To use this module, type:\\
\begin{pycode}
import pm.log.filters
\end{pycode}

\subsection{Default filters}
\begin{description}
 \item [RemoveImmediateRepetitionsFilter] Removes immediate repetitions of activities (only one activity per
    repetition row is left).
 \item [CaseLengthFilter] Keeps the cases whose length satisfy the conditions specified in the
    constructor (above and/or below some thresholds.
 \item [PrefixerFilter] Prefixes each activity with some given prefix. (Yes, I know this should be an encoder rather than a filter\ldots)
 \item [FrequencyFilter] Only the cases above a given frequency threshold (in number of occurrences or as a percentage) are kept.
\end{description}

\subsection{Filtering}
The main function is the \texttt{filter\_log} function, that takes a log and a filter as parameters. In the following example, we use a frequency filter to retain the most frequent cases so that the amount of the log preserved is over 35\%.\\
\begin{pycode}
>>> log = pm.log.Log(cases=5*[['A','B','A','B']]+3*[['C','D']]+2*[['A','C','B']])
>>> log.get_uniq_cases()
defaultdict(<type 'int'>, {('A', 'C', 'B'): 2, ('C', 'D'): 3, ('A', 'B', 'A', 'B'): 5})
>>> ff = pm.log.filters.FrequencyFilter(log, log_min_freq=0.35)
#All unique cases with at least 5 cases will be kept
#Filter will save 5 cases (50.0%)
#Filter will save 1 unique sequences (33.3%)
>>> flog = pm.log.filters.filter_log(log, ff)
>>> flog.get_uniq_cases()
defaultdict(<type 'int'>, {('A', 'B', 'A', 'B'): 5})
\end{pycode}

In this example, keeping the most frequent case (a 33\% of all unique cases), we already preserve 50\% of the log (considering repetitions).

\section{Clustering}
To use the clustering algorithms, import the module as follows:\\
\begin{pycode}
import pm.log.clustering
\end{pycode}

This module offers the following clustering algorithms:
\begin{description}
 \item [random\_balanced\_clusters] Returns a list of logs obtained by random
    balancing clustering. Actually the order is not random but follows the original log order. To obtain a random one, you can use the \texttt{shuffle} function in the standard random module:\\
\begin{pycode}
>>> import random
>>> random.shuffle(log.get_cases(True))
\end{pycode}
% 
% \begin{Verbatim}[commandchars=\\\{\}]
% \PY{o}{>>}\PY{o}{>} \PY{k+kn}{import} \PY{n+nn}{random}
% \PY{o}{>>}\PY{o}{>} \PY{n}{random}\PY{o}{.}\PY{n}{shuffle}\PY{p}{(}\PY{n}{log}\PY{o}{.}\PY{n}{get\PYZus{}cases}\PY{p}{(}\PY{n+nb+bp}{True}\PY{p}{)}\PY{p}{)}
% \end{Verbatim}
This clustering method works with plain and enhanced logs (i.e., it returns enhanced logs).
 \item [max\_alphabet\_clusters] Creates $n$ clusters (logs). The first $n-1$ contain cases such that the
    alphabet of the log is below the given threshold. The last one contains all 
    remaining cases.
 \item [optional\_clusters] Returns $n$ clusters (logs) such that they all have a size less than the given threshold by 
    splitting cases that contain and do not contain a particular activity. The
    activity chosen is the one that gives the more balanced splitting. If no
    splitting activity is available, then the cluster is not further split, 
    regardless of its size.
 \item [xor\_activities\_clusters] Splits in clusters selecting a pair of mutually exclusive activities.
    Produces three logs: one containing all the cases in which the first 
    exclusive activity appears, the second containing all the cases of the 
    second activity, and the last one containing all the cases in which none of
    the previous activities appears.
\end{description}

\section{Noise}
Import the noise module with:\\
\begin{pycode}
import pm.log.noise
\end{pycode}

This module only includes a single function, \texttt{log\_from\_noise} that takes a log and a probability of error and generates a new (plain) log with suppressions/insertions/swaps of activities. Currrently there are two noise types: \emph{single} and \emph{maruster}. Refer to the docstrings.

An example with 50\% of error probability: \\
\begin{pycode}
>>> l.get_cases()
[['a', 'b', 'd', 'c', 'f'],
 ['a', 'c', 'b', 'd', 'f'],
 ['a', 'c', 'd', 'b', 'f'],
 ['a', 'd', 'e', 'f'],
 ['a', 'b', 'c', 'd', 'f'],
 ['a', 'e', 'd', 'f']]
>>> nl = pm.log.noise.log_from_noise(l, noise='single', perror=0.5)
>>> nl.get_cases()
[['a', 'b', 'd', 'c'],
 ['a', 'c', 'b', 'd', 'f'],
 ['a', 'c', 'd', 'b', 'f'],
 ['a', 'd', 'd', 'e', 'f'],
 ['a', 'b', 'c', 'd', 'f'],
 ['a', 'e', 'd', 'f']]
\end{pycode}

\chapter{Transition Systems}
To work with Transition Systems (TSs), the ts module has to be imported:\\
\begin{pycode}
import pm.ts
\end{pycode}

\section{Obtainig a TS from a log}
The most important function in this module is \texttt{ts\_from\_log} that generates a TS from a log by applying some conversion method.
The current conversion methods are: \textit{seq} (sequential), \textit{set}, \textit{mset} (multiset), and \textit{cfm} (common final marking). See \cite{SoleC10} for the conversion details. Additionally there are a couple of parameters, \emph{tail} and \emph{folding} that expect integers. Tail forces to consider only the last $n$ activities seen in the case, while folding applies folding strategies to simplify Petri net discovery \cite{SoleCfoldings}. For instance:\\

\begin{pycode}
>>> ts = pm.ts.ts_from_log(l, 'mset', folding=1)
\end{pycode}

will yield a TS in which states have been folded in a way that the Petri net found using 1-bounded regions is exactly the same as if we used the sequential or multiset conversion without foldings.

The conversion is actually performed by the log2ts tool, so if this tool is not available this function will not work.

\section{Loading/Saving}
A TS can be saved using the \emph{save} function of the class:\\

\begin{pycode}
>>> ts.save('ts.sg')
\end{pycode}

The default format is the SIS format, although other formats are available. However, be careful since currently we are only providing loading capabilities for the SIS format, through this function:\\

\begin{pycode}
>>> ts = pm.ts.ts_from_file('ts.sg')
\end{pycode}

\section{Manual modification of TSs}
We currently provide only methods to add states and edges. Example:\\
\begin{pycode}
>>> ts = pm.ts.ts_from_file('f96q3.set.sg')
>>> s4 = ts.add_state('s4')
>>> ts.add_edge('s3','added',s4)
#<Edge object with source '2' and target '4' at 0xaf2bf14>
\end{pycode}

This yields the TS in Figure~\ref{fig:ts}. Note that we can reference vertices using the vertex descriptor or by name.
\begin{figure}[htb]
 \centering
 \includegraphics[height=5cm]{ts.pdf}
 \caption{TS with an added vertex and transition.}
 \label{fig:ts}
\end{figure}


There are some technical problems with removing vertices\footnote{graph\_tool reorders all vertex descriptors when one of them is removed, invalidating our dictionary that maps state names to these descriptors.}, thus the preferred mechanism is to filter states and edges.

To filter edges and vertices, use the graph\_tool set\_edge\_filter and set\_vertex\_filter on the graph member of the TS (with name \emph{g}). That is:\\

\begin{pycode}
 ts.g.set_vertex_filter(bmap)
\end{pycode}

where bmap is a Boolean property map containing True for the vertices to be kept. Please refer to the graph\_tool documentation (\url{http://projects.skewed.de/graph-tool/doc/index.html}).

\section{Visualization}
For visualization we rely on third-party libraries/applications. First of all, the graph\_tool package offers a drawing function that can either use cairo or graphviz. To these two options we also added the possibility to call the draw\_astg application that comes with Petrify (also expected to be in \texttt{/usr/local/bin}).

To draw a TS, simply call the draw function on the object. Some examples:\\

\begin{pycode}
>>> ts.draw('ts.png')
>>> ts.draw('ts_astg.ps','astg')
>>> ts.draw(None, 'cairo')
\end{pycode}

The first version writes a PNG file using graphviz (the default driver). The second uses the draw\_astg application to generate a PS file. The last one is used for interactive output. However, in this last version, the labels of the edges are lost, so it is of limited utility (also the nodes tend to overlap). Valid output formats include PS, PDF, SVG and PNG. The output format is deduced from the filename extension. Clearly this part needs some cleaning and modifications, but this is what we have by the moment.

\section{Operations related to logs}

A useful operation is to incorporate frequency information to the TS states and arcs. For that purpose there exists the \texttt{map\_log\_frequencies} function. The frequency information is graphically depicted in the \texttt{draw} function by means of larger nodes and wider edges (when using the cairo and graphviz drivers, not the astg). Let us see an example:\\
\begin{pycode}
>>> log.get_uniq_cases()
defaultdict(<type 'int'>, {('N', 'N'): 46, ('L',): 41, 
			  ('L', 'L'): 146, ('L', 'L', 'L', 'L'): 64})
>>> ts.map_log_frequencies(log, state_freq=True, edge_freq=True, state_cases=True)
\end{pycode}

The resulting TS is shown in Figure~\ref{fig:ts_freq}. As you can see the labelling algorithm of graph\_tool is not perfect, since some labels are hidden by the width of the edges, but there is a graphical representation of the number of sequences visiting each vertex and edge.
\begin{figure}[htb]
 \centering
 \includegraphics[height=8cm]{ts_freq.pdf}
 \caption{TS with state and edge frequencies.}
 \label{fig:ts_freq}
\end{figure}
In this particular example we have also instructed the algorithm to store the numbers of the unique sequences going through each state. The frequency can be retrieved using the \texttt{get\_state\_frequency} method, while the list of unique sequences is obtained via the \texttt{get\_state\_cases} method:\\
\begin{pycode}
>>> ts.get_state_frequency('s5')
46
>>> ts.get_state_cases('s5')
array([0])
>>> log.get_uniq_cases().items()[0]
(('N', 'N'), 46)
\end{pycode}


Some functions related to logs and TSs have been compiled into a special module:\\
\begin{pycode}
import pm.log_ts
\end{pycode}

Currently there is only a single function, namely \texttt{cases\_included\_in\_ts} that returns a log containing only the cases (and part of the cases) that fit inside the TS. Additionally, some related information is also printed in the screen:\\

\begin{pycode}
>>> ml = pm.log_ts.cases_included_in_ts(l, ts)
#Total cases in log:  7
#Excluded:            0   0.00%
#Partially Included:  1  14.29% average included length 50.00%
#Totally Included:    6  85.71%
\end{pycode}

\section{Taking advantage of the algorithms in graph\_tool}

The graph\_tool package offers a wide range of algorithms on graphs. The member \emph{g} of a TS object contains the underlying graph that can be passed as parameter to any of the graph\_tool algorithms. For instance if you want to obtain the vertices of the TS \emph{ts} (in this particular case this corresponds to Figure~\ref{fig:ts_freq}) sorted in topological order, you can use:\\
\begin{pycode}
>>> import graph_tool.all as gt
>>> sort = gt.topological_sort(ts.g)
>>> print sort
[5 4 3 1 6 2 0]
\end{pycode}
Note that these values are the internal vertex indices, so if you want to obtain the corresponding state name, you must first obtain the vertex object, and then the name of that vertex:\\ 
\begin{pycode}
>>> [ts.vp_state_name[ts.g.vertex(i)] for i in sort]
['s4', 's3', 's2', 's1', 's6', 's5', 's0']
\end{pycode}

\chapter{Petri Nets}
The Petri Nets module is imported with:\\
\begin{pycode}
import pm.pn
\end{pycode}

\section{Obtaining a PN from a TS}
The discovery algorithms can be accessed through the \texttt{pn\_from\_ts} function. The current methods are based on the theory of regions. This function receives as parameters the discovery method to use (rbminer or stp), the desired $k$-boundedness and an aggregation factor.\\

\begin{pycode}
>>> pn = pm.pn.pn_from_ts(ts, k=2, agg=4)
\end{pycode}

For instance in this example the rbminer tool is used (the default one), to search for 2-bounded regions with an aggregation factor of 4. This latter factor is ignored when using stp, which searches for ALL minimal regions. Since no simplification is performed afterwards, it can generate more places.

\subsection{It takes eons to finish, what can I do?}
Some advices: if using rbminer, reduce the aggregation factor. Depending on the number of state of the TS and the number of activities, smaller values might be needed to avoid a combinatorial explosion. Values from 2 to 4 often work. 

Sometimes the problem is not the aggregation factor, but the simplification step when lots of places have been found. Rbminer includes an option to disable the simplification (currently not handled in the python function). However it is usually better to try with other approaches, since if the simplification phase could not finish due to the large number of places, non-simplifying the net will yield an uncomprehensible model anyway.

\section{Loading/Saving}
A PN can be saved using the \texttt{save} method, as usual. The target format is the SIS format.\\

\begin{pycode}
>>> pn.save('test.g')
>>> pn2 = pm.pn.pn_from_file('test.g')
\end{pycode}

The function \texttt{pn\_from\_file} allows loading a saved PN. It infers the format from the filename extension, unless explicitely provided. Two input formats are currently supported: SIS (.g extension) and PNML (.pnml extension).

\section{Visualization}
Works similarly to the TS visualization. However we currently provide only support for the cairo and draw\_astg drivers (simply we have not the time yet to write the code to handle the graphviz driver). When using draw\_astg the valid output formats are PS, GIF and DOT.

\section{Simulation}
The PN simulation has its own module:\\
\begin{pycode}
import pm.pn.simulation
\end{pycode}

This module contains a single function, \texttt{simulate}, that returns a log in which all cases where obtained by simulating the PN with the given parameters.\\

\begin{pycode}
>>> slog = pm.pn.simulation.simulate(pn,method='rnd',num_cases=10,length=4)
>>> slog.get_cases()
[['a', 'b', 'd', 'c'],
['a', 'c', 'b', 'd'],
['a', 'e', 'd', 'f'],
['a', 'b', 'd', 'c'],
['a', 'c', 'b', 'd'],
['a', 'c', 'b', 'd'],
['a', 'e', 'd', 'f'],
['a', 'c', 'b', 'd'],
['a', 'c', 'b', 'd'],
['a', 'd', 'c', 'b']]
\end{pycode}

Depending on the parameters, please be patient, this code has not been optimized for performance.

\chapter{Causal Nets}
The Causal Nets module is imported with:\\
\begin{pycode}
import pm.cnet
\end{pycode}

\section{Obtaining a Cnet from a log}
The easiest way to obtain a C-net that can replay the log is to use the immediately follows C-net. The function that computes this trivial C-net is \texttt{immediately\_follows\_cnet\_from\_log}:\\

\begin{pycode}
>>> cif = pm.cnet.immediately_follows_cnet_from_log(clog)
\end{pycode}

For more interesting C-nets we use an SMT solver, called stp. The stp solver is a little bit picky about variable names. Moreover, logs to be used for C-net discovery have to satisfy some properties related to the initial and final activities. Thus, this module provides a function that conditions a given log for C-net discovery.\\

\begin{pycode}
>>> clog = pm.cnet.condition_log_for_cnet(l)
\end{pycode}

Once a suitable log is available, we can use the \texttt{cnet\_from\_log} function. With the default parameters, it returns a tuple whose first element is the discovered C-net, and the second one is an object containing the binding frequencies computed. The discovered C-net has been arc minimized, and then all redundant bindings have been removed. A skeleton (i.e., a list of activity pairs) can be provided also. The binding frequencies can be later on used for enhancing model representation (particularly BPMN diagrams derived from the C-net).

A nice way to obtain a reasonable skeleton is to use the heuristics of the Heuristic Miner. The function \texttt{flexible\_heuristic\_miner} provides this functionality:\\

\begin{pycode}
>>> skeleton = pm.cnet.flexible_heuristic_miner(clog,0.6,0.6,0.5)
>>> cn, bfreq = pm.cnet.cnet_from_log(clog, skeleton=skeleton)
\end{pycode}

\section{Loading/Saving}
As with other objects in the suite, the Cnet class has its own save method. The formats available are JSON (native) or ProM (in two variants, the new and the old format). At the time of writing this document with ProM6.1 the format is still the old one.\\

\begin{pycode}
>>> cn.save('test.cn')
>>> cn.save('test.prom.cnet',format='prom')
>>> cn.save('test.oldprom.cnet',format='prom_old')
\end{pycode}

Loading is provided through the following module function:\\

\begin{pycode}
>>> cn = pm.cnet.cnet_from_file('test.cn')
\end{pycode}

\section{Visualization}
C-nets are complex to depict. We provide a prototype of interactive visualizer based on the pygame package. Do not expect it to work correctly for large C-nets, however it will do the trick for the small ones (this is better than nothing). Just call the draw function to start the visualizer.\\

\begin{pycode}
>>> cn.draw()
\end{pycode}

\section{C-net manipulation}
C-nets can be merged simply by using the addition operator.\\

\begin{pycode}
>>> cnm = cn + cif
\end{pycode}

\section{Simulation}

The simulation functions are located inside the simulation module of this sub-package:\\
\begin{pycode}
import pm.cnet.simulate_cnet
\end{pycode}

\begin{description}
 \item [fitting\_cases\_in\_skeleton] Returns a log containing the sequences that would fit in any C-net
    using the arcs in [skeleton] (regardless of the actual bindings that would be required). Useful if using a skeleton generated using the FHM and you do not want to add elements to the skeleton to include all the behavior of the log, but want to retain as much as the behavior of the log that can be explained by that skeleton.
\item [nonfitting\_cases\_in\_cn] Returns a log with all the cases of the given log that are non-fitting in the given C-net.
\item [fitness] Returns the fitness of the given log with respect to the C-net.
\end{description}


\section{Operations related to Transition Systems}
We can compute a lower bound on the ETC metric for a given TS using the \texttt{etc\_lower\_bound} method. In the following example we develop a complete case from the beginning:\\

\begin{pycode}
>>> import pm.log
>>> log = pm.log.log_from_file('f96q3.tr')
>>> import pm.cnet
>>> clog = pm.cnet.condition_log_for_cnet(log)
>>> ts = pm.ts.ts_from_log(clog, 'mset')
>>> ts.map_log_frequencies(clog)
>>> cn, bfreq = pm.cnet.cnet_from_log(clog)
>>> cn.etc_lower_bound(ts)
0.902352418996893
\end{pycode}

Here we first load a log, we condition it for C-net discovery and we build a TS using the sequential conversion (we must use this conversion to check the ETC metric) to whom we add the frequency information. Finally, we pass this TS to the \texttt{etc\_lower\_bound} function that returns a lower bound on the ETC metric. In this particular example the C-net is a very precise model, with a lower bound of 90.2\%.

This value is a lower bound because the algorithm does not actually check if the escaping edges are actually valid C-net cases (they could not, in which case the real number of escaping edges would be inferior, thus the real ETC value would be larger).

\chapter{BPMN diagrams}
The BPMN module is imported with:\\
\begin{pycode}
import pm.bpmn
\end{pycode}

\section{Obtaining a BPMN diagram from a Cnet}
The following example show how to obtain a BPMN from a C-net derived from a log. \\

\begin{pycode}
>>> import pm.cnet
>>> log = pm.log.Log(cases=[['A','B'],['C','D']])
>>> clog = pm.cnet.condition_log_for_cnet(log)
>>> cn, bfreq = pm.cnet.cnet_from_log(clog)
>>> import pm.bpmn
>>> bp = pm.bpmn.bpmn_from_cnet(cn)
>>> bp.add_frequency_info(clog, bfreq)
Checking A
Checking C
Checking B
    with E: 1 100.0% 50.0%
Checking E
Checking D
    with E: 1 100.0% 50.0%
Checking S
    with C: 1 50.0% 100.0%
    with A: 1 50.0% 100.0%
\end{pycode}

Note that the BPMN is enhanced depicting (through the arc width in the diagram) the frequency of each path according to the binding frequency obtained during the C-net computation, and that the log must also be provided. The output gives tabular information: for instance, the line\\
\begin{pycode}
Checking B
    with E: 1 100.0% 50.0%
\end{pycode}
indicates that there is a single trace through the arc connecting B and E, and that this single sequence represents the 100\% of the sequences that go out from B and the 50\% of the sequences that arrive to E.

The BPMN class provides also another method to enhance the output that is related to the durations of the activities. In particular, the \texttt{add\_activity\_info} function takes a dictionary that maps each activity to a dictionary containing information of the activity. If this dictionary contains the entry \emph{avg\_duration}, then the activities are depicted using a color scheme that represents fast activities (with respect to the average duration of an activity) with green colors and slow activities with red colors.

To generate a DOT file containing the BPMN diagram, use the \texttt{print\_dot} function. If additional information has been provided through some enhancement function, it will be automatically displayed.

\section{Manually creating a BPMN diagram}

It is perfectly possible to create a BPMN diagram from scratch or to add arcs/activities to an existing one. Use the \texttt{add\_element} and \texttt{add\_connection} functions:\\

\begin{pycode}
from pm.bpmn import BPMN, Activity, Gateway

b = BPMN()
act1 = b.add_element(Activity('hola'))
b.add_connection(b.start_event, act1)
act2 = b.add_element(Activity('adeu'))
b.add_connection(act2, b.end_event)
gw = b.add_element(Gateway('exclusive'))
x1 = b.add_element(Activity('exclusive1'))
x2 = b.add_element(Activity('exclusive2'))
b.add_connection(act1, gw)
b.add_connection(gw, ['exclusive1','exclusive2'])
gw2 = b.add_element(Gateway('exclusive'))
b.add_connection(['exclusive1','exclusive2'], gw2)
b.add_connection(gw2,'adeu')
b.print_debug()
b.print_dot('bpmn.dot')
!dot -Tps 'bpmn.dot' > 'bpmn.ps'
\end{pycode}

This code generates the following figure:
\begin{figure}[htp]
 \centering
 \includegraphics[height=7cm]{bpmn-crop.pdf}
 \caption{BPMN diagram}
 \label{fig:bpmn}
\end{figure}

To generate a PDF from the PS file, use the following commands:
\begin{Verbatim}
ps2pdf bpmn.ps
pdfcrop bpmn.pdf
\end{Verbatim}

\section{Loading/Saving}

Currently, there are no load/save operations for this class, so you have to generate it from the C-net every time (yes, I know, this is something we must work out at some point) or just use the Dot file generated.

%\section{Simulation}
\chapter{Adding packages/modules to the suite}

To add a package \emph{foo}, one should simply create a directory inside the directory where the pm package is located, and inside the foo directory, create a \_\_init\_\_.py file. Then this package can be imported as\\
\begin{pycode}
import pm.foo
\end{pycode}

Similarly, if one wants to create a module \emph{bar} inside this package, then a \emph{bar.py} file has to be create inside the foo directory. Then the module can be imported with\\
\begin{pycode}
import pm.foo.bar
\end{pycode}

However, my recomendation is that you create your scripts on your own folders, simply importing the modules you need from the suite. These scripts can quite easily become a package or a module later on. The idea is that if you feel your scripts add necessary/useful functionality, you can send them to us, so that we can merge them with the ``official'' code. Note that the current package distribution stores the code in terms of the objects they work on (logs, TSs, PNs, etc.). In general this is a reasonable option, since techniques that work well for all kinds of models are, unfortunately, not very frequent.

Please try to follow as much as possible the conventions in the code:
\begin{itemize}
 \item The function that creates an object of type X from another of type Y is usually named \texttt{X\_from\_Y}.
 \item Similarly, to retrieve X from a file (load) we usually have a function \texttt{X\_from\_file}.
 \item The function to save an object is name \texttt{save}.
 \item The function to obtain a graphical representation of an object is named \texttt{draw}, although we usually keep this name if the function accepts similar parameters to the already existing draw methods.
\end{itemize}

\bibliographystyle{alpha}
\bibliography{ProcessDiscovery}
 
\end{document}
